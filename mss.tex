\documentclass[twoside]{article}

\usepackage{ustj}

\addbibresource{mss.bib}

\newcommand{\authorname}{Stuart Christoph}
\newcommand{\authorpatp}{\patp{lagrev-nocfep}}
\newcommand{\affiliation}{\patp{tocwex} Syndicate}

%  Make first page footer:
\fancypagestyle{firststyle}{%
\fancyhf{}% Clear header/footer
\fancyhead{}
\fancyfoot[L]{{\footnotesize
              %% We toggle between these:
              Manuscript submitted for review.\\
              % {\it Urbit Systems Technical Journal} I:1 (2024):  1–3. \\
              ~ \\
              Address author correspondence to \authorpatp.
              }}
}
%  Arrange subsequent pages:
\fancyhf{}
\fancyhead[LE]{{\urbitfont Urbit Systems Technical Journal}}
\fancyhead[RO]{Azimuth and Decentralized Identity}
\fancyfoot[LE,RO]{\thepage}

%%MANUSCRIPT
\title{Ruminations on the Utility of Azimuth and Decentralized Identity}
\author{\authorname~\authorpatp \\ \affiliation}
\date{}

\begin{document}

\maketitle
\thispagestyle{firststyle}

\begin{abstract}
\end{abstract}

% We will adjust page numbering in final editing.
\pagenumbering{arabic}
\setcounter{page}{1}

\tableofcontents

\section{Introduction}

The nature of identity has captivated human attention since prehistory. Cave painting self-portraits and handprints reveal an early sense of `self' in even the earliest hominids. More recently, Descartes' {\emph cogito, ergo sum} laid the groundwork for modern philosophy. As we move into the digital age, the concept of `identity' continues to evolve in order to extend into online networks.

This article considers the concept of identity within the context of Urbit and its identity layer, Azimuth. We explore how these systems implement decentralized identity, examining their technical foundations, practical applications, and the broader implications for digital self-sovereignty. By understanding these elements, we aim to shed light on the future of identity in a decentralized world.

\subsection{Decentralized Identity}

Urbit isn't the only project working on the concept of decentralized identity. As a starting point, we will examine two perspectives on decentralized identity:

\begin{enumerate}
  \item Self-Sovereign Identity
  \item Decentralized Identifiers
\end{enumerate}

Blockchain developer Christopher Allen coined the term `self-sovereign identity' (\textsc{ssi}) to concretize a notion of digitial decentralized identity built for individual control and self-determination \citep{Allen2016}. He noted that the evolution of online identity has passed through eras of centralized administrative control, oligarchic federated systems, and unfortunately, institutionally captured ``user-centric'' designs. The next era, designated the era of \textsc{ssi}, thankfully aims to return us to a more intuitive understanding of digital identity. Allen defined ten key principles for \textsc{ssi}: \footnote{The appendix contains an excerpt defining the principles more completely}.
% Allen2016 https://www.lifewithalacrity.com/article/the-path-to-self-soverereign-identity/

\begin{tabular}{ll}
  \begin{itemize}
    \item  Existence
    \item  Control
    \item  Access
    \item  Transparency
    \item  Persistence
  \end{itemize}
  &
  \begin{itemize}
    \item  Portability
    \item  Interoperability
    \item  Consent
    \item  Minimalization
    \item  Protection
  \end{itemize}
\end{tabular}

However, we contend that each of the pre-\textsc{ssi} eras has critically pulled us away from the sense of `self' that is vital to identity–the way we experience our identity in the world of atoms and the digital experiences which are becoming ingrained in our day-to-day lives.

Separate from Allen's principles, W3C has presented a standard for Decentralized Identifiers (\textsc{did}s). According to ``Decentralized Identifiers (\textsc{did}s) v1.0", in part \citep{W3C2022}:
% (ref https://www.w3.org/TR/did-core/)

\begin{quote}
[Decentralized identifiers](https://www.w3.org/TR/did-core/#dfn-decentralized-identifiers) (\textsc{did}s) are a new type of identifier that enables verifiable, decentralized digital identity. A [\textsc{did}](https://www.w3.org/TR/did-core/#dfn-decentralized-identifiers) refers to any subject (e.g., a person, organization, thing, data model, abstract entity, etc.) as determined by the controller of the [\textsc{did}](https://www.w3.org/TR/did-core/#dfn-decentralized-identifiers).
\ldots{}
While other parties might be used to help enable the discovery of information related to a [\textsc{did}](https://www.w3.org/TR/did-core/#dfn-decentralized-identifiers), the design enables the controller of a [\textsc{did}](https://www.w3.org/TR/did-core/#dfn-decentralized-identifiers) to prove control over it without requiring permi\textsc{ssi}on from any other party. [\textsc{did}s](https://www.w3.org/TR/did-core/#dfn-decentralized-identifiers) are [URIs](https://www.w3.org/TR/did-core/#dfn-uri) that associate a [\textsc{did} subject](https://www.w3.org/TR/did-core/#dfn-did-subjects) with a [\textsc{did} document](https://www.w3.org/TR/did-core/#dfn-did-documents) allowing trustable interactions associated with that subject.
\end{quote}

While the Urbit project may not traditionally rely on `Earth standards,' the \textsc{did} recommendation also defines ten design goals that are informative for understanding decentralized identity. (FOOTNOTE: See section 6: Appendix for full excerpt defining the design goals):

\begin{tabular}{ll}
  \begin{itemize}
    \item  Decentralization
    \item  Control
    \item  Privacy
    \item  Security
    \item  Proof-based
  \end{itemize}
  &
  \begin{itemize}
    \item  Discoverability
    \item  Interoperability
    \item  Portability
    \item  Simplicity
    \item  Extensibility
    \end{itemize}
\end{tabular}

We will revisit some of these principles and design goals in the context of Azimuth and Urbit, but first, a reminder to developers: defaults matter.

\subsection{Defaults Matter}

As \patp{wicdev-wisryt} noted of the public key infrastructure (\textsc{pki}) idea maze (ref pcmonk piece), the long history of \textsc{pki}s largely splits between being `decentralized and unused' or `centralized and easy to subject oneself to'.  \patp{wicdev-wisryt} explored this idea maze along the branches of global consistency, permanence, and self-ownership, to which we add a fourth branch: sane defaults.

\textsc{pgp}, \textsc{gpg}, and Web of Trust solutions have existed for decades but are used by effectively nobody, and in some sense can be dismissed due to a lack of sane defaults. One could argue the shortcoming of these systems is a lack of global consistency, but in practice global consistency is a necessary but not sufficient condition for a self-sovereign identity. If the system does not also provide sane defaults the global consistency remains an illegible mess to the typical user. The Web of Trust and similar solutions may be glossed over as demonstrably insufficient.

If we consider global consistency, permanence, and self-ownership as useful goals for the \textsc{pki}, we must recognize that they fail to explain some of the more controversial elements of Azimuth's hierarchical model for some critics. In fact, alternate conceptions for \textsc{pki}s and identity more broadly are being explored by both Kinode\footnote{As of writing, Kinode's \texttt{.os} identity implementation leans towards allowing users to bring their own identity system, and handling packet routing by designated and self-declared routing nodes.} and Plunder.\footnote{As of writing, Plunder has not implemented an identity system and in the author's conversations with developers working on the project, aims to be unopinionated about either identity system or packet routing and peer discovery, instead deferring that role to projects building on top of the system.} These projects aim to satisfy each of the three branches but take different approaches to the concept of sane defaults. So why does Urbit's Azimuth retain a hierarchical address space?

Aside from being a clean way of naming a finite address space, Azimuth's hierarchy provides a \emph{sane default} for packet routing, peer discovery, and third-party service provision. It gives users a starting point, and a set of alternatives that are tractable. It serves as the crisis moment (`It's dangerous to go alone! Take this') before embarking on the journey of self-sovereign identity.

Some critics frame this structure as `digital feudalism' but fail to understand two other elements of the system:
\begin{enumerate}
  \item  The default \textsc{pki} includes a permi\textsc{ssi}onless right to exit, requiring no approval from the network node up the hierarchy from you.
  \item  Even the hierarchical model is \emph{just a sane default} and does not forbid an alternative.
\end{enumerate}

\noindent
While no alternative exists at present, there is no reason that a particular Urbit ship must exclusively ask its sponsor for addresses of its peers. Nor is there any technical restriction on Urbit's routing packets using a different model. In summary, \emph{the hierarchy is a sane default in an otherwise infinite game.}

\section{The Current State of Azimuth}

We understand the need to place the \textsc{pki} on a distributed ledger with global consensus and recognize the hierarchical model as a sane default. But what do we know about the current address space and its technical implementation? Since 2017, Azimuth has been on hosted on the Ethereum blockchain, and in 2022 the contracts were updated to support a ``naive'' Layer 2 rollup. Returning to the principles and design goals of \textsc{ssi} and \textsc{did}s, this on-chain presence provides benefits such as existence, control, transparency, persistence, decentralization, and discoverability. It also supports many of the other stated goals, if not satisfying them purely through an on-chain presence. Many of these principles were either explicitly or implicitly considered with both the initial Azimuth contracts (ref blog post from midden fabler), and the move to supporting the Layer 2 rollup (ref blog post about solving the gas crisis). However, the current state of Azimuth falls short in terms of providing interoperability and extensibility. 

\subsection{Layer 1 IDs on the Ethereum network}

The deployment of Azimuth on Ethereum was one of the first uses of the EIP-721 standard (ref https://eips.ethereum.org/EIPS/eip-721). It is implemented across two core contracts: Azimuth, which holds the registry of identity ownership, and Ecliptic, which defines the business logic for updating the state of the registry. To this day, these contracts serve as one of the most technically `useful' ERC-721s on the Ethereum network, powering the peer-to-peer interactions of thousands of nodes (also known as urbit ships) in an encrypted off-chain network. As an ERC-721, Urbit ID experienced a significant increase in valuation during the NFT boom of 2021, with the market cap of address space breaking \$2B at its peak.

However, the core downside of Azimuth's presence on Ethereum should not be understated." In exchange for the benefits of NFT hype cycles, Urbit users must pay a tax to the Ethereum network in the form of gas fees. This interaction is particularly expensive because the original versions of Ecliptic required all state transition computations to occur on-chain. While this means Azimuth inherits the security benefits of Ethereum's global consensus model, there is a contention that it fundamentally extracts value from the Urbit network and deposits it into Ethereum in an unnecessary exchange.

\subsection{Layer 2 IDs on the naive rollup}

The transaction costs on Ethereum were so great that, at times, it cost over \$200 worth of ETH just to get on the network. For a budding distributed system project, such costs can be unbearable. Thus, the `solution to the gas crisis' (ref https://urbit.org/blog/rollups) was introduced: a custom `naive' rollup. The key feature of the naive rollup is that, instead of paying Ethereum miners (now validators) to verify the state transitions of Azimuth, each Urbit node can validate those transitions themselves. This again brings us back to the idea of sane defaults in a decentralized or distributed network. 

That is, in running your urbit, the canonical ``Azimuth" is nothing more than a sane default. This is exemplified by the fact that you could run a `fakezod' network locally and act as any Urbit ID you chose, or even deploy an alternative version of Azimuth onto mainnet Ethereum. However, because everyone else is likely operating on the default version of `azimuth.hoon`, which assumes `azimuth.eth` as The One True \textsc{pki}, any of your potential peers will discard your packets as invalid. The `Urbit network' will ignore you. Whether or not the Urbit network pays the Ethereum network for state transition computation, it remains true that Urbit's off-chain interactions are voluntary and governed by sane defaults.

\subsection{When to Use Global Consensus}

As noted in `~wicdev-wisryt`'s piece, blockchains allow us to distinguish between two cryptographically valid messages, specifically which was signed first, without requiring a centralized party. Both the Layer 1 and Layer 2 implementations rely on this affordance. Effectively, they both need the chain to give them a timestamp and a signed message to discern key ownership. Whether the computation of key rotation is done on-chain or client side, anyone within the Urbit network that is continuing to use the sane defaults the system provides will ultimately get the correct result for validating a packet's sender. It is the core affordance of timestamped messages for which Urbit must pay *some* blockchain to have a viable decentralized \textsc{pki}.

However, Urbit-side key exchange and peer discovery are not the only benefits of having the \textsc{pki} on Ethereum. The cost of on-chain computation for global consensus is undoubtedly a hurdle to low-cost user adoption of Urbit. However, it would be a failure of imagination to claim there is no contingent loss of benefits. Specifically, Ethereum is a growing network that is building on top of the same ERC-721 standard on which Urbit ID is defined. Improvements to the capabilities of the Ethereum network can enhance the capabilities of the Urbit network—*if Urbit maintains, or increases, its legibility to the Ethereum Network*.  The current naive rollup does not achieve this. An L2 Urbit ID, and it's ownership in particular, is functionally illegible to the outside world. From the perspective of Ethereum, global consistency is compromised, and key rotation is impo\textsc{ssi}ble to validate. 

While we could explore the technical details of these two systems, instead let's return to examine the \textsc{ssi} principles and \textsc{did} design goals of interoperability and extensibility. In a sense, these just mean that you want your identity to work in as many places as po\textsc{ssi}ble. To be as legible to as many systems or stakeholders as is po\textsc{ssi}ble. 

A playful example of this is whether you would trade your state-issued driver's license for a high school ID card (ref https://sarlev-sarsen.rooftopdao.io/blog/why-l1masterrace-is-more-than-a-meme). In spite of your driver's license being more costly to replace or update, your response was almost certainly, ``No way." You know intuitively that your driver's license is a more credible and legible identifier than a high school ID. Now think if you could achieve that level of credibility and legibility in a self-sovereign manner.

But this tradeoff isn't just a fun thought experiment; a dependency on centralized identity issuance authorities has real world impacts:

> ``In the last year, self-sovereign identity has also entered the sphere of _international policy_ [17](https://blogs.worldbank.org/en/digital-development/world-citizen-transforming-statelessness-global-citizenship). This has largely been driven by the refugee crisis that has beset Europe, which has resulted in many people lacking a recognized identity due to their flight from the state that issued their credentials. However, it’s a long-standing international problem, as foreign workers have often been abused by the countries they work in due to the lack of state-issued credentials." (ref https://www.lifewithalacrity.com/article/the-path-to-self-soverereign-identity/)

Now, we can ask, `If our Urbit IDs are going to be the foundation for a self-sovereign digital identity, do we want them to be more legible or less legible?'

While this may run counter to some Urbit purists who want to avoid anything related to `Earth code,' the reality for the foreseeable future is that interoperability with other networks is both desirable and necessary for Urbit's continued survival. With time, as Urbit evolves, it may become the most legible system, and the cruft of Earth code will fall by the wayside. But that future is still a long way off.

Improving legibility, or in other words, protecting the interoperability and extensibility of Urbit ID, does not imply loyalty to a particular chain. It simply claims that increasing legibility to the largest set of networks should be included in the objective function of Urbit ID.

\section{Alternative Options for the \textsc{pki}}

\How do we continue to improve Azimuth? While in 2017 the options for the \textsc{pki} were limited, the technology landscape has drastically evolved in the intervening years. 

\subsection{Self-Host the \textsc{pki} on \lstinline[style=inlinecode]{%chain}}
\
The most `philosophically straightforward' answer would be to self-host the \textsc{pki}. While this requires further technical development, it is conceptually tractable and the \lstinline[style=inlinecode]{%chain} project by `~tiller-tolbus` and `~midsum-salrux` has shown initial successes on an urbit-native blockchain. This option could make use of identity-backed consensus models, and would be the most direct pathway to the Urbit Network not having to pay a `tax' to another network for its consensus mechanism. In a sense, this is the most `independent' or `self-contained' option. However, it would likely result in decreased capital access and practical legibility to external networks.

While reasonable minds may differ on the net benefit of Urbit's proximity to the Ethereum network's capital pools, there is no doubt that such proximity has had material impacts on address space valuation. This includes the \textasciitilde\$33k peak for star-level identities. Distancing Urbit ID from Ethereum by moving the \textsc{pki} to a self-hosted network would limit the opportunity to access this capital by increasing the barriers to exchange.

Moving the \textsc{pki} in this way would also decrease interoperability and extensibility by making it harder for the Ethereum network to access and validate data about ownership of Urbit IDs. Technically speaking, this constraint could be reduced by self-hosted RPC endpoints of \lstinline[style=inlinecode]{%chain} or Azimuth state that could be accessed over HTTP. However, the more prominent issue would be the coordination constraint it would generate. If the \textsc{pki} moves, what motivation do members of the Ethereum Network have to *bother* with Urbit RPC endpoints instead of just staying within the Ethereum tooling ecosystem?

In a world where Urbit is more powerful and offers something compelling to Ethereum, you could see the draw. However, the ~70k per year in gas fees that Urbit currently pays to maintain the \textsc{pki} on Ethereum seems worth the legibility it affords.

\subsection{Zenith:  UrbitChain on Cosmos \texsc{sdk}}

Of course, moving the \textsc{pki} to a self-hosted, urbit-native solution isn't the only option if the Urbit Network wants to reduce its dependency on outside networks. For example, there are ongoing discu\textsc{ssi}ons about an `Urbitchain,' governed by the galaxy-level address space and implemented as its own chain using the Cosmos \textsc{sdk}.

In terms of legibility, this pathway essentially splits the difference. The Cosmos ecosystem is smaller than Ethereum as a whole—coming in at #2 on metrics for Total Value Locked (TVL) (ref https://thedefiant.io/news/markets/cosmos-tvl-surge-ethereum) —but could give some of the benefits of a self-hosted system. For example, it would reduce the outflow of capital from the Urbit network, while still achieving some interoperability with the external world thanks to the affordances granted by Cosmos. Peg zones and the Inter-blockchain Communication (IBC) protocol would mean access to a network of blockchain networks. These networks are likely to have more operational friction for capital flows than something like remaining on L1 Ethereum, but may be a valid tradeoff.

\subsection{Split the \textsc{pki} across multiple chains}

As we consider the tradeoffs of moving the \textsc{pki}, it is worth revisiting our `sane defaults' branch in the idea maze. If `azimuth.eth` and with it `azimuth.hoon`, are just sane defaults, why not also support some sane *alternatives*? One of the final Urbit precepts (ref Urbit precepts) from  `~wicdev-wisryt` is `Communities are autonomous.'

To this end, we should use Urbit OS as the coordination technology to allow users and communities to decide with which other networks they desire to maintain legibility. As with the current \textsc{pki}'s split between L1 Ethereum and the naive rollup, so too could we allow people to move their Urbit ID from L1 Ethereum to any other chain with a viable way to post signature data. 

Some may choose to remain on mainnet Ethereum to take advantage of features like ERC-6551 Token Bound Accounts(ref https://eips.ethereum.org/EIPS/eip-6551) (FOOTNOTE: EIP-6551 implements an interface and registry for smart contract accounts deterministically controlled by your ERC-721 Urbit ID). Others may want to inscribe their ownership as an Ordinal on Bitcoin. Alternatively, Solana offers some benefits for the more crypto-degenerate who are looking for speed of trading and market volatility to help make their wealth. 

Considering this path, it is worthwhile to elaborate on how the current L2 solution works. The original version of the Ethereum contracts held ownership data in the Azimuth contracts and business logic in the Ecliptic contracts. In 2021, these contracts were updated to support essentially two new affordances:

\begin{itemize}
  \item  Azimuth points sent to the ethereum deposit address (`0x1111...`) are assessed by the default Urbit client as being owned on layer 2.
  \item  Ecliptic includes a function (TK, check where this function actually lives? Open questions out to `~norsyr-torryn`) for posting *time-stamped* signature data to the Ethereum chain concerning the change of ownership of a point owned on layer 2, and the default urbit client will review and validate the state change before applying the change to the urbit's own `azimuth.hoon`.
\end{itemize}

This mechanism means that these points are still cryptographically controlled for purposes of anyone running the default Urbit OS or relevant code (namely, `eth-watcher.hoon`, `naive.hoon`, `azimuth-tracker.hoon` and `azimuth.hoon`), and it is easy to imagine how this same mechanism could be used to support other chains as hosts for portions of the \textsc{pki}:
1. Define a holding address for the ERC-721 on mainnet ethereum for other host chains (i.e. Sending the L1 NFT to `0x2222...` indicates sending the ownership to bitcoin)
2. Implement a watcher for the subject chain that looks for posted signature data
3. Update and/or add to the relevant Urbit OS code for client side validation of data posted to the new chain.
This approach would increase legibility to other chains, giving communities more options for what their members can achieve with their Urbit ID. 

However, as the \textsc{pki} fractures, maintaining global consistency becomes increasingly complex. In the above model, the primary solution for maintaining this consistency becomes that official `global consensus' is on mainnet Ethereum, while any other alternative is subject only to the client side validation. Depending on the affordances of the subject chain, the level of effort put into building the subsystem, and any cost concerns for network fees on the specific subject chain, the result is likely that *moving to any other chain is a one-way trip*. Just as a move to the current naive rollup is a one way trip.

The biggest reason for this being the case is the question: `what timestamp is the canonical one?'. 

\subsection{Move the \textsc{pki} to Nockchain}

While Ethereum has made significant progress in interchain bridges and L2 rollups since the implementation of the Naive rollup, making two-way bridges increasingly po\textsc{ssi}ble, they still introduce a host of security and consensus questions. They also are likely to exacerbate the capital outflow concerns of the current L1 Ethereum implementation.

But if we return to the philosophical perspective, assuming we want our identities to be legible to different networks, don't we also want to modify that legibility? Just as individuals can emigrate between nation-states, it seems self-evident that we should be able to move between digital networks, bringing our self-sovereign identities with us.

What is the best way to enable this portability (another shared principle/design goal for \textsc{ssi}s/\textsc{did}s) while maintaining global consistency? If you read the section title, and know what Nockchain is, then you will know what I am about to say:

ZK Proofs.

Using Zorp's EDEN and associated Nock prover (ref https://eprint.iacr.org/2023/1021), any Urbit could prove valid computation of any `client-side' Azimuth state change and use that proof as the data to post to any given blockchain. It is likely that the `easiest' place to do that would be Nockchain. However, external legibility might incentivize these proofs to be posted elsewhere. An Azimuth contract on a subject chain could use the proof of valid nock computation to a\textsc{ssi}gn ownership of an Urbit ID crypto-asset native to a given address. This would include whatever affordances native asset ownership on that chain may bestow to an Urbit operator.

In the long run, this pathway would likely lead more Urbit operators to move their Urbit ID's primary `citizenship' to Nockchain. This would occur without unnecessarily losing legibility to other networks and with the po\textsc{ssi}bility of absorbing external capital into the Urbit network along the way. 

\section{Other Considerations for Azimuth as a decentralized identity system}

While Azimuth's on-chain nature is that of a public key infrastructure, a \textsc{pki} on it's own doesn't make for a complete identity system. In fact, it is the pairing of Urbit ID with Urbit OS that presents the opportunity to serve as a full-fledged decentralized identity system. Whether we look at implementing W3C's standardized \textsc{did}s, or adhering to the principles of \textsc{ssi}s, not everything about Azimuth is, or should be, on-chain.

\subsection{Illegible subnets}

Globally available and transparent chain state greatly enhances legibility. Openly available sponsorship hierarchies inform unconnected peers about how to reach you, and global claims can be used for various purposes. However, not everything needs global consensus (ref [[How much consensus? How smart of contracts?]]). In fact, many networks may want to be illegible to the outside. Private access control lists, go\textsc{ssi}p networks, and packet routing mechanisms enable use cases that would not benefit from the cost or transparency of on-chain or globally consistent legibility. Urbit's off-chain and encrypted peer-to-peer messaging affordances open up the landscape of `illegible subnets' while still benefiting from associations with specific pseudonymous urbit identities.

The value of illegible subnets also builds on the insight that not everything can, or should, be distilled down to its lowest common denominator, tokenized, and put on the blockchain. The more complex the interaction, the more costly the distillation process—and *the more likely there is no consensus to be found around its `valuation'*. Instead of demanding timestamped global consensus in all our peer interactions, we must support emergent, implied-consensus networks. These networks are governed not by immutable chain state but by willing association and participation by sovereign peers. Systems where the network forms and dissolves on the fly, based on the ongoing discovery and refinement of the values of its participants. 

It is worth noting that illegible subnets may contend with the \textsc{ssi} principle of consent, in that they enable opaque data-sharing networks and other similar `darknet' use cases.

\subsection{Peer-to-peer attestations}

Another affordance that does not require a blockchain is peer-to-peer attestation. There are many use cases where the privacy and cost benefits of cryptographically verifiable peer-to-peer attestations are desirable. They can be used independently, ``~sarlev-sarsen attests to having received a message from ~lagrev-nocfep containing a valid passcode", or even in conjunction with global blockchain state.

For example, \%fund from ~tocwex.syndicate (ref tocwexsyndicate.com?), uses peer-to-peer attestation begin to build a subjective reputation graph. An Urbit can hold a signed message from an Ethereum wallet stating, ``~sarlev-sarsen controls wallet `0x6789...`" and can even share this signed message with other peers. Anyone receiving that attestation can then use it in constructing an understanding of the relationships between different Urbit IDs, off-chain financial promises, on-chain transactions, or other economic activity, without it being a canonical, globally available understanding.

(TK, maybe insert some code from `\%fund-profile` here?)

This sharing of the understandings granted by peer-to-peer attestations can thus create pockets of privileged information networks, and the attendant competitive advantages of a more opaque, yet verifiable, system.
#### 4.3 Self-verified credentials
As `~wicdev-wisryt` notes, one offshoot of the permanence branch of the \textsc{pki} idea maze makes available a mechanism for self-attestation. This currently only exists for the top-level of Azimuth's address space where comet identities reside. This self-attestation is useful for a variety of purposes—cost, anonymity, etc.—but doesn't afford a more long-lasting system for self-verified credentials due to the transient nature of a comet identity. It is likely that expanding the capacity for self-attestations would be of benefit to the Urbit network. In particular, the concept of Verifiable Credentials (VC) (ref https://www.w3.org/TR/vc-data-model/) may be worth pursuing.

Urbit ID does not currently have a system of verifiable credentials, and arguably, the \textsc{pki} aspect of Azimuth should never include this. However, it should be reasonably straightforward to produce either a VC attestation layer utilizing Azimuth or incorporate knowledge of Azimuth into a VC solution.  

One mechanism that could be integrated into an Urbit-native VC solution is the Urbit HD (Hierarchical Deterministic) Wallet. This is the technology used to secure ownership of an Urbit ID point and could satisfy the \textsc{ssi} doctrinal requirement that, “an \textsc{ssi} digital wallet should implement open standards for portable, self-sovereign verifiable credentials and other sensitive private data” (ref Preukschat & Reed). Without VC, Urbit does not yet qualify for \textsc{ssi} under this definition.  


\section{Conclusion}

Much work remains to be done on both the on-chain implementations and off-chain affordances of Azimuth and the Urbit ID system. Each area of work must be considered alongside the effort required for implementation:

\begin{itemize}
  \item  Balancing optionality in the \textsc{pki} for end users against the ongoing maintenance of a more complex codebase.
  \item  Weighing legibility for external networks against measures that reduce capital outflows from the network.
  \item  Allocating development time to adherence with \textsc{did}, \textsc{ssi}, and VC standards against the practical usefulness of their implementation.
  \item  Coordinating systems for illegible subnets and peer-to-peer attestations against other features that can be productized and marketed to consumers.
\end{itemize}

Regardless of the specific focus, Azimuth and Urbit ID offer developers a platform to build systems that enhance individual sovereignty and provide flexibility in creating community networks that control their own legibility and interactions. \tombstone{}

\section{Appendix}

This appendix serves as a reference for the more complete excerpts of referenced articles.

\subsection{The Path to Self-Sovereign Identity}

The principles of \textsc{ssi} are as follows. An excerpt from the section titled ``Ten Principles of Self-Sovereign Identity," published April 26, 2016:

\begin{quote}
  A number of different people have written about the principles of identity. Kim Cameron wrote one of the earliest “Laws of Identity”[20](https://www.lifewithalacrity.com/article/the-path-to-self-soverereign-identity/#fn:20), while the aforementioned Respect Network policy[21](https://www.lifewithalacrity.com/article/the-path-to-self-soverereign-identity/#fn:21) and W3C Verifiable Claims Task Force FAQ[22](https://www.lifewithalacrity.com/article/the-path-to-self-soverereign-identity/#fn:22) offer additional perspectives on digital identity. This section draws on all of these ideas to create a group of principles specific to self-sovereign identity. As with the definition itself, consider these principles a departure point to provoke a discu\textsc{ssi}on about what’s truly important.

  These principles attempt to ensure the user control that’s at the heart of self-sovereign identity. However, they also recognize that identity can be a double-edged sword — usable for both beneficial and maleficent purposes. Thus, an identity system must balance transparency, fairness, and support of the commons with protection for the individual.

  \begin{enumerate}
    \item **Existence.** _Users must have an independent existence._ Any self-sovereign identity is ultimately based on the ineffable “I” that’s at the heart of identity. It can never exist wholly in digital form. This must be the kernel of self that is upheld and supported. A self-sovereign identity simply makes public and acce\textsc{ssi}ble some limited aspects of the “I” that already exists.
    \item **Control.** _Users must control their identities._ Subject to well-understood and secure algorithms that ensure the continued validity of an identity and its claims, the user is the ultimate authority on their identity. They should always be able to refer to it, update it, or even hide it. They must be able to choose celebrity or privacy as they prefer. This doesn’t mean that a user controls all of the claims on their identity: other users may make claims about a user, but they should not be central to the identity itself.
    \item **Access.** _Users must have access to their own data._ A user must always be able to easily retrieve all the claims and other data within his identity. There must be no hidden data and no gatekeepers. This does not mean that a user can necessarily modify all the claims associated with his identity, but it does mean they should be aware of them. It also does not mean that users have equal access to others’ data, only to their own.
    \item **Transparency**. _Systems and algorithms must be transparent._ The systems used to administer and operate a network of identities must be open, both in how they function and in how they are managed and updated. The algorithms should be free, open-source, well-known, and as independent as po\textsc{ssi}ble of any particular architecture; anyone should be able to examine how they work.
    \item **Persistence.** _Identities must be long-lived._ Preferably, identities should last forever, or at least for as long as the user wishes. Though private keys might need to be rotated and data might need to be changed, the identity remains. In the fast-moving world of the Internet, this goal may not be entirely reasonable, so at the least identities should last until they’ve been outdated by newer identity systems. This must not contradict a “right to be forgotten”; a user should be able to dispose of an identity if he wishes and claims should be modified or removed as appropriate over time. To do this requires a firm separation between an identity and its claims: they can’t be tied forever.
    \item **Portability.** _Information and services about identity must be transportable._ Identities must not be held by a singular third-party entity, even if it’s a trusted entity that is expected to work in the best interest of the user. The problem is that entities can disappear — and on the Internet, most eventually do. Regimes may change, users may move to different jurisdictions. Transportable identities ensure that the user remains in control of his identity no matter what, and can also improve an identity’s persistence over time.
    \item **Interoperability.** _Identities should be as widely usable as po\textsc{ssi}ble._ Identities are of little value if they only work in limited niches. The goal of a 21st-century digital identity system is to make identity information widely available, cro\textsc{ssi}ng international boundaries to create global identities, without losing user control. Thanks to persistence and autonomy these widely available identities can then become continually available.
    \item **Consent.** _Users must agree to the use of their identity._ Any identity system is built around sharing that identity and its claims, and an interoperable system increases the amount of sharing that occurs. However, sharing of data must only occur with the consent of the user. Though other users such as an employer, a credit bureau, or a friend might present claims, the user must still offer consent for them to become valid. Note that this consent might not be interactive, but it must still be deliberate and well-understood.
    \item **Minimalization.** _Disclosure of claims must be minimized._ When data is disclosed, that disclosure should involve the minimum amount of data necessary to accomplish the task at hand. For example, if only a minimum age is called for, then the exact age should not be disclosed, and if only an age is requested, then the more precise date of birth should not be disclosed. This principle can be supported with selective disclosure, range proofs, and other zero-knowledge techniques, but non-correlatibility is still a very hard (perhaps impo\textsc{ssi}ble) task; the best we can do is to use minimalization to support privacy as best as po\textsc{ssi}ble.
    \item **Protection.** _The rights of users must be protected._ When there is a conflict between the needs of the identity network and the rights of individual users, then the network should err on the side of preserving the freedoms and rights of the individuals over the needs of the network. To ensure this, identity authentication must occur through independent algorithms that are censorship-resistant and force-resilient and that are run in a decentralized manner.
  \end{enumerate}
\end{quote}

\subsection{Decentralized Identifiers (\textsc{did}s) v1.0}

The goals of the design of \textsc{did}s are as follows. An Excerpt from Section 1.2 from version 1.0 of the W3C Recommendation 19 July 2022:

\begin{quote}
[Decentralized Identifiers](https://www.w3.org/TR/did-core/#dfn-decentralized-identifiers) are a component of larger systems, such as the Verifiable Credentials ecosystem [[VC-DATA-MODEL](https://www.w3.org/TR/did-core/#bib-vc-data-model ``Verifiable Credentials Data Model v1.1")], which influenced the design goals for this specification. The design goals for Decentralized Identifiers are summarized here.

|Goal|Description|
|---|---|
|Decentralization|Eliminate the requirement for centralized authorities or single point failure in identifier management, including the registration of globally unique identifiers, public verification keys, [services](https://www.w3.org/TR/did-core/#dfn-service), and other information.|
|Control|Give entities, both human and non-human, the power to directly control their digital identifiers without the need to rely on external authorities.|
|Privacy|Enable entities to control the privacy of their information, including minimal, selective, and progre\textsc{ssi}ve disclosure of attributes or other data.|
|Security|Enable sufficient security for requesting parties to depend on [\textsc{did} documents](https://www.w3.org/TR/did-core/#dfn-did-documents) for their required level of assurance.|
|Proof-based|Enable [\textsc{did} controllers](https://www.w3.org/TR/did-core/#dfn-did-controllers) to provide cryptographic proof when interacting with other entities.|
|Discoverability|Make it po\textsc{ssi}ble for entities to discover [\textsc{did}s](https://www.w3.org/TR/did-core/#dfn-decentralized-identifiers) for other entities, to learn more about or interact with those entities.|
|Interoperability|Use interoperable standards so [\textsc{did}](https://www.w3.org/TR/did-core/#dfn-decentralized-identifiers) infrastructure can make use of existing tools and software libraries designed for interoperability.|
|Portability|Be system- and network-independent and enable entities to use their digital identifiers with any system that supports [\textsc{did}s](https://www.w3.org/TR/did-core/#dfn-decentralized-identifiers) and [\textsc{did} methods](https://www.w3.org/TR/did-core/#dfn-did-methods).|
|Simplicity|Favor a reduced set of simple features to make the technology easier to understand, implement, and deploy.|
|Extensibility|Where po\textsc{ssi}ble, enable extensibility provided it does not greatly hinder interoperability, portability, or simplicity.|
\end{quote}

\selectlanguage{USenglish}
\printbibliography
\end{document}
